%!TEX TS-program = xelatex
%!TEX encoding = UTF-8 Unicode
% Awesome CV LaTeX Template for CV/Resume
%
% This template has been downloaded from:
% https://github.com/posquit0/Awesome-CV
%
% Author:
% Claud D. Park <posquit0.bj@gmail.com>
% http://www.posquit0.com
%
% Template license:
% CC BY-SA 4.0 (https://creativecommons.org/licenses/by-sa/4.0/)
%


%-------------------------------------------------------------------------------
% CONFIGURATIONS
%-------------------------------------------------------------------------------
% A4 paper size by default, use 'letterpaper' for US letter
\documentclass[11pt, a4paper]{awesome-cv}

% Configure page margins with geometry
\geometry{left=1.4cm, top=.8cm, right=1.4cm, bottom=1.8cm, footskip=.5cm}

% Color for highlights
% Awesome Colors: awesome-emerald, awesome-skyblue, awesome-red, awesome-pink, awesome-orange
%                 awesome-nephritis, awesome-concrete, awesome-darknight
\colorlet{awesome}{awesome-red}
% Uncomment if you would like to specify your own color
% \definecolor{awesome}{HTML}{3E6D9C}

% Colors for text
% Uncomment if you would like to specify your own color
% \definecolor{darktext}{HTML}{414141}
% \definecolor{text}{HTML}{333333}
% \definecolor{graytext}{HTML}{5D5D5D}
% \definecolor{lighttext}{HTML}{999999}
% \definecolor{sectiondivider}{HTML}{5D5D5D}

% Set false if you don't want to highlight section with awesome color
\setbool{acvSectionColorHighlight}{true}

% If you would like to change the social information separator from a pipe (|) to something else
\renewcommand{\acvHeaderSocialSep}{\quad\textbar\quad}


%-------------------------------------------------------------------------------
%	PERSONAL INFORMATION
%	Comment any of the lines below if they are not required
%-------------------------------------------------------------------------------
% Available options: circle|rectangle,edge/noedge,left/right
% \photo[rectangle,edge,right]{./examples/profile}
\photo[circle, noedge, left]{./examples/cat_1.png}
\name{Chun-Hao}{Chang}
\position{National Cheng Kung Universityr{\enskip\cdotp\enskip}B.S. degree in Electrical Engineering}
% \address{235, World Cup buk-ro, Mapo-gu, Seoul, 03936, Republic of Korea}

\mobile{(+886) 0902-257-959}
\email{e24096409@gs.ncku.edu.tw}
%\dateofbirth{January 1st, 1970}
% \homepage{www.posquit0.com}
\github{Haouo}
\linkedin{chun-hao-chang-686719218}
% \gitlab{gitlab-id}
% \stackoverflow{SO-id}{SO-name}
% \twitter{@twit}
% \skype{skype-id}
% \reddit{reddit-id}
% \medium{madium-id}
% \kaggle{kaggle-id}
% \googlescholar{googlescholar-id}{name-to-display}
%% \firstname and \lastname will be used
% \googlescholar{googlescholar-id}{}
% \extrainfo{extra information}

% \quote{``Be the change that you want to see in the world."}


%-------------------------------------------------------------------------------
\begin{document}

% Print the header with above personal information
% Give optional argument to change alignment(C: center, L: left, R: right)
\makecvheader[C]

% Print the footer with 3 arguments(<left>, <center>, <right>)
% Leave any of these blank if they are not needed
\makecvfooter
{\today}
{Chun-Hao Chang~~~·~~~Résumé}
{\thepage}


%-------------------------------------------------------------------------------
%	CV/RESUME CONTENT
%	Each section is imported separately, open each file in turn to modify content
%-------------------------------------------------------------------------------
%-------------------------------------------------------------------------------
%	SECTION TITLE
%-------------------------------------------------------------------------------
\cvsection{Summary}


%-------------------------------------------------------------------------------
%	CONTENT
%-------------------------------------------------------------------------------
\begin{cvparagraph}

    %---------------------------------------------------------
    I am now purchasing B.S. degree of Electrical Engineering in National Cheng Kung University. I am also a professional cat lover! I keep a cat I encounter on the street. Besides, I joined Stray Animal Volunteer Team in our school and became one of the committee member after one year. Hence, I am absolutely sure that I love animals especial cats and dogs!
\end{cvparagraph}


\lettersection{Why Tomofun?}
\begin{cvparagraph}
    I have great interesting in AI and its applications on IoT devices. I have taken many courses related like "AI Robotics" and "AI Computing Architecture and System".

    Hence, I think this internship will be an awesome opportunity for me to learn how to design such a great and animal-friendly IoT devices with the power of Machine Learning.
\end{cvparagraph}


%-------------------------------------------------------------------------------
%	SECTION TITLE
%-------------------------------------------------------------------------------
\cvsection{Education}


%-------------------------------------------------------------------------------
%	CONTENT
%-------------------------------------------------------------------------------
\begin{cventries}

  %---------------------------------------------------------
  \cventry
  {B.S. degree in Electrical Engineering} % Degree
  {National Cheng Kung University} % Institution
  {Tainan, Taiwan} % Location
  {Sep. 2020 - Jun. 2024} % Date(s)
  {
    \begin{cvitems} % Description(s) bullet points
      \item {Intern Student in AI System Lab, NCKU EE}
      \item {Teaching Assistant in PlayLab, NCKU SOC}
      \item {Member of Swimming Team}
    \end{cvitems}
  }

  \cventry
  {Regular Class}
  {National Changhua Senior High School}
  {Changhua, Taiwan}
  {Sep. 2017 - Jun. 2020}
  {
    \begin{cvitems}
      \item {Member of Swimming Team}
    \end{cvitems}
  }

  %---------------------------------------------------------
\end{cventries}

%-------------------------------------------------------------------------------
%	SECTION TITLE
%-------------------------------------------------------------------------------
\cvsection{Skills}


%-------------------------------------------------------------------------------
%	CONTENT
%-------------------------------------------------------------------------------
\begin{cvskills}
  \cvskill
  {Programming Language} % Category
  {C, C++, Python} % Skills

  \cvskill
  {Hardware Design Language}
  {Verilog, Chisel}

  \cvskill
  {EDA tools}
  {HSPICE, Virtuoso, Laker}

  \cvskill
  {Others}
  {Linux Environment, Git}
\end{cvskills}

%-------------------------------------------------------------------------------
%	SECTION TITLE
%-------------------------------------------------------------------------------
\cvsection{Work Experience}


%-------------------------------------------------------------------------------
%	CONTENT
%-------------------------------------------------------------------------------
\begin{cventries}

  %---------------------------------------------------------
  \cventry
  {Teaching Assistant of AI Computing Architecture and System, Spring 2023} % Job title
  {Playlab} % Organization
  {NCKU SOC} % Location
  {Feb. 2023 - Present} % Date(s)
  {}

  \cventry
  {Teaching Assistant of AI Computing Architecture and System, Fall 2022}
  {Playlab}
  {NCKU SOC}
  {Sep. 2022 - Jan. 2023}
  {}

\end{cventries}

%-------------------------------------------------------------------------------
%	SECTION TITLE
%-------------------------------------------------------------------------------
\cvsection{Courses Related}


%-------------------------------------------------------------------------------
%	CONTENT
%-------------------------------------------------------------------------------
\begin{cventries}
  \cventry
  {Spring 2023}
  {Advanced Computer Architecture and AI Chip Design}
  {NCKU SOC}
  {}
  {OoO CPU, RISC-V Pipeline CPU implementation and topics about Neural Network Optimization}

  \cventry
  {Spring 2023}
  {VLSI Digital Circuit Design}
  {NCKU EE}
  {}
  {VLSI Digital Design concepts and techniques such as speed \& power optimization, circuit simulation with HSPICE and layout design with Laker}

  \cventry
  {Spring 2023}
  {Computer Algorithm}
  {NCKU EE}
  {}
  {Basic algorithms, including Sorting, Dynamic Programming, Greedy Algorithm, Graph Algorithms and advanced Data Structures}

  \cventry
  {Fall 2022} % time
  {Computer Organization} % course name
  {NCKU EE}
  {} % remains blank
  {Basic Computer Architecture concepts with RISC-V ISA} % description

  \cventry
  {Fall 2021 - Fall 2022}
  {Microelectronics}
  {NCKU EE}
  {}
  {Semiconductors, BJTs and MOSFET, Small Signal Model and different types of Amplifiers}

  \cventry
  {Spring 2022}
  {AI Computing Architecture and System}
  {NCKU SOC}
  {}
  {Simple RISC-V CPU implementation and basic ideas about how to accelerate AI Computing by SIMD instruction and Systolic Array}

  \cventry
  {Fall 2021 - Spring 2022}
  {Electric circuits}
  {NCKU EE}
  {}
  {Electric circuit theory}

  \cventry
  {Spring 2021}
  {Logical System}
  {NCKU EE}
  {}
  {Basic concepts about Logical Design and Digital Circuits}

  \cventry
  {Fall 2020 - Spring 2021}
  {Introduction to Computers}
  {NCKU EE}
  {}
  {Programming concepts and techniques like Object-Oriented with C++ Programming Language}
\end{cventries}

% \input{resume/honors.tex}
% \input{resume/certificates.tex}
% \input{resume/presentation.tex}
% \input{resume/writing.tex}
% \input{resume/committees.tex}
% \input{resume/extracurricular.tex}


%-------------------------------------------------------------------------------
\end{document}
